\section*{Conclusiones}
{
    \hspace{0.5cm} La m\'ascara de red es muy importante en la red ya que asegura que la direcci\'on sea \'unica en la red.
    La manera en que se comporta la red desde capa 3 es mucho m\'as compleja de la manera en que lo hace desde capa 2, pues 
    aqu\'i hay m\'as dispositivos que administrar y m\'as configuraciones que llevar a cabo. La m\'ascara de red asegura que 
    se esten usando los segmentos de red adecuadamente y que no hayan conflictos con otros dispositivos en la misma.\newline

    \hspace{0.5cm} Configurar las redes en el router fu\'e el paso m\'as importante por que si bien es cierto algunos dispositivos
    tenian comunicaci\'on a nivel de capa 2, no pod\'ian comunicarse con el otro segmento de la red. Configurar las interfaces de red
    fu\'e el objetivo m\'as importante de la pr\'actica.\newline

    \hspace{0.5cm} En el ambiente f\'isico no fu\'e posible terminar esta pr\'actica, es por eso que la misma fu\'e llevada al entorno
    de simulaci\'on donde se pudieron afinar algunos detalles y corregir los errores que se cometieron en el entorno f\'isico.
}

